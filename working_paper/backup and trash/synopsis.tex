\documentclass{article}

\title{Synopsis: Shifts in the Beveridge curve and propensity to post}

\author{Niels-Jakob Harbo Hansen and Hans Henrik Sievertsen}

\begin{document}

\section{Motivation}

\begin{itemize}
	\item In many countries there recently have appeared to be shifts in the Beveridge curve. 
	\begin{itemize}
		\item Insert graphs - especially for European countries as these did not use surveys before 2008 (Elsby et al., forthcoming)
	\end{itemize}
	\item Some have taken this as evidence for a decreased matching efficiency (cite: e.g. Draghi Jackson hole speech, OECD).
	\item However, a different possibility is change in employers propensity to post vacancies.
	\begin{itemize}
		\item A point mentioned by some (e.g. Elsby et al., forthcoming) but not yet properly investigated.
	\end{itemize}
	\item Specific contribution
\end{itemize}

\section{Related literature and contribution}

\begin{itemize}
	\item Davis, Faberman and Haltiwanger (2013).
	\item Read through references in Elsby, Michaels and Rattner (forthcoming) for references.
	\item Contribution
	\begin{itemize}
	\item First to investigate firms propensity to hire via PES postings.
	\item First to investigate how various unemployed uses various channels.
	\end{itemize}
	
\end{itemize}

\section{A simple illustrative model}

\begin{itemize}
	\item Our model simplified: an \emph{exogenous} share of announced vacancies and an \emph{exogenous} fraction of time spend searching on this market.
	\item Show how vacancy share shifts the observed Beveridge curve in this model. 
	\item Take away: changes in the fraction announced can lead to shifts in the calibrated Beveridge curve.
\end{itemize}


\section{Data sources and creation}

\begin{itemize}
	\item Explain: Data sources
	\item Explain: Data creation
	\item Explain: Definition of announced and unannounced hire.
\end{itemize}

\section{Patterns in the cross section and across time}

\begin{itemize}
	\item Firm characteristics
	\begin{itemize}
		\item Sector
		\item Industry
		\item Wage deciles
		\item Size
		\item Growth
	\end{itemize}
	\item Worker characteristics
		\begin{itemize}
		\item Education
		\item Previous position
		\item Experience
		\item ....
    \end{itemize}
	\item Share of hires via announced openings over time
	\begin{itemize}
		\item How much of this is caused by compositional change in the cross section?
	\end{itemize}
\end{itemize}

\section{Implications}

\begin{itemize}
	\item How big are the implied shifts in the BC?
\end{itemize}




\end{document}
